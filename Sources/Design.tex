\chapter{Analysis and design}
One of the objectives of this work was the creation of a tool for processing and visualization of results from TPM devices. Such a tool already exists, and it is called \texttt{JCAlgTest process}. However, the tool's current design makes it difficult to extend it. Extensive refactoring could help in gaining more extensibility. Nevertheless, a complete rework proved more effective in this case.

This chapter describes the original tool for generating visualizations and discusses its problems. Then solutions to those problems are proposed, and it is described how they have been used in the reimplementation.

\section{JCAlgTest}\label{sec:jcalgtest}
The \texttt{JCAlgTest}\footnote{\url{https://github.com/crocs-muni/JCAlgTest}} is a suite of tools used for automated analysis of cryptographic smartcards, specifically those running the JavaCard platform. It takes advantage of the officially available JavaCard API to test for support of specific algorithms, measures their performance characteristics, and collects general information about the tested smart card. The suite consists of three modules:
\begin{itemize}
  \item
        \texttt{JCAlgTestJavaCard} is a JavaCard applet that needs to be uploaded onto the tested smart card. According to instructions from \texttt{JCAlgTestClient} application to execute code for specified operation. Firstly it tries to instantiate a particular object belonging to the operation. If it succeeds, that means that the operation is supported, and in the case of performance testing, it can proceed further with the execution. If the instantiation was unsuccessful, a specific exception is thrown, which usually means that the operation is not supported.
  \item
        \texttt{JCAlgTestClient} is a host application which is responsible for communication with \texttt{JCAlgTestJavaCard} applet via APDU commands and responses. The information about the algorithm execution time is measured and recorded. The JavaCard platform does not support any time measurement methods. The execution time measurement during the performance testing is estimated externally due to the limitations of the JavaCard platform, which does not support any time measurement methods.\\
  \item
        \texttt{JCAlgTestProcess} is an application that is used for the visualization of JavaCard and TPM results. Results in the form of CSV files are processed into various kinds of visualizations and tabular data. As the state of the tool is deemed obsolete, its reimplementation was needed and was performed as one of the goals of this thesis. The reasons behind the reimplementation and the new design are discussed in the following sections.
\end{itemize}

\section{JCAlgTest process}
The \texttt{JCAlgTest process} is an application used for tabular and chart visualisations of JavaCard smart card and TPM 2.0 run time data created by \texttt{JCAlgTest} and \texttt{tpm2-algtest} tools respectively. 

This section will describe the application's current state and its problems.
\subsection{Current state}
%\todo{Add diagram visualising basic parts of the old app}

The application is written in Java programming language. It is designed mainly as a command-line tool for generating HTML visualizations for JavaCard smart card run time data. Still, it also contains means of visualizing TPM 2.0 run time data. The project's website shows JavaCard smart card visualizations \cite{jcalgtestWeb}. Each type of visualization is performed by building an HTML file and either by embedding JavaScript calls to visualization functions contained in  \texttt{D3.js library} or \texttt{Google Charts API} or by creating a basic HTML table with added CSS styling and usability improvements such as sorting or filtering of entries using JavaScript. 

An \texttt{Apache Ant} command-line tool is used to build the application. It uses XML files to describe the build processes such as compiling, running, or testing. It is also supported by many programming IDEs, most notably Netbeans IDE or IntelliJ IDEA.

The \texttt{JCAlgTest process} provides a command-line API. The users can specify the input, output directory, and the type of visualization they want to generate.

\subsection{Problems}\label{subsec:design-problems}
The whole \texttt{JCAlgTest} suite was developed over several years by various people. An initial version of \texttt{JCAlgTest process} along with the rest of \texttt{JCAlgTest} suite was created by doc. Petr Švenda, my supervisor. Following that, Rudolf Kvašňovský further developed the \texttt{JCAlgTest process} tool as a part of his bachelor's thesis~\cite{Kvasnovsky2016thesis}. From there on, several more minor updates from various people were performed. The application serves its intended purpose well, however, any extensions to it require lots of code analysis and patience so that nothing breaks and no new issues arise.

One of the problems of the application is the lack of object representation of device profiles. The data is primarily stored in simple Java array lists or maps. That is why a relatively simple task of extending functionality may prove to be very difficult. 

Another problem is the amount of hard-coded HTML tags into strings. One unintentional mistake of deleted angle bracket may corrupt the whole HTML output. To change, for instance, something in the metadata of each generated HTML, it is needed to manually change the string, which creates the body part of the HTML document. 

\section{Improvements}
In this section, the improvements for the \texttt{JCAlgTest process} tool will be discussed along with how they have been applied to its reimplementation.

\subsection{Object design}
As mentioned in \myref{Section}{subsec:design-problems}, the object design of the tool needed some improvements. The new design should provide classes mapping the information contained in input files. They will be referred to as device profiles. The device stands for either TPM 2.0 or JavaCard smart card. The data is in the form of CSV files. However, the format of some files does not conform to any specific implementation for CSV format, as there is no header for the data, and the count of delimiters is variable on each line. This effectively results in the need to parse files using custom-made parser procedures instead of relying on library functions.

\subsubsection{Device profile types}\label{sec:device-profiles}
There are three main types of device profiles:
\begin{enumerate}
    \item \textbf{Support} In the case of TPM devices, it contains information about the support of specific algorithms, commands, or structures defined in the TPM 2.0 specification. However, in the case of JavaCard smart cards, the information is about supported cryptographic algorithms defined in JavaCard API. Along with this information, it also contains basic information about the devices. This information ranges from vendor information to memory limits.
    
    \item \textbf{Performance} Contain run time benchmarks for cryptographic algorithms. Similarly, as in the previous case, the tested algorithms, commands, or structures originate from either TPM 2.0 specification for TPMs, or JavaCard API in the case of JavaCard smart cards. In the case of JavaCard smart cards, the performance results are divided into fixed and variable results. Variable-length performance results measure data length dependant performance, whereas fixed-length performance results measure it with static data length depending on a particular algorithm.
    
    \item \textbf{Cryptographic Properties} They contain information about the execution time duration for cryptographic algorithms supported by the TPM, additionally, they record the key pairs generated using RSA or ECC algorithms for analysis. The information is split into multiple data sets, and each data set contains only specific information about the tested TPM. The TPM is mentioned specifically because cryptographic properties in this form are recorded only for TPM devices.
\end{enumerate}

In the implementation, the object design of these profiles is very straightforward. The classes were modeled according to the information contained in the CSV files. Their objective is only to store the data so that it can be later manipulated. That is why there was no need for any complex abstractions in design. However, even such basic improvement by creating a class representation makes it easier to work with the data and the resulting code more readable compared to using array lists and indexes.

The profile classes are very similar. They all have to store specific results and other information inherited from their abstract base class. The measurement results are stored in dictionaries where we associate the identification of a result with its object or a list of these objects for easier checking if a specific result was measured.

The only exception to this system is the representation of profiles of cryptographic properties. These do not have a designed object form, but most of them are in the form of well-formed CSV files, so they can be manipulated using library functions.


\subsection{Building outputs}
Another problem to resolve was to find a way to build HTML outputs in a way where a small mistake does not result in a corrupted file and hours of debugging. One solution is a templating engine where we prepare a special file very similar to pure HTML with the exception of the special syntax that takes care of populating the template with data. This is the way more often used in modern front-end frameworks such as React or Vue. But templating may not always be the best idea, because it steepens the learning curve for understanding the code. That is because everyone needs to understand the new syntax used in the templates. The solution used in the reimplementation was the use of a library called \texttt{dominate} which makes us able to create HTML documents and fragments using pure Python language. Additionally, it was used in another project which also works with JavaCards called \texttt{scrutiny}, which was created as a part of a master's thesis by Imrich Nagy~\cite{Nagy2019thesis}. The reimplementation also uses some code from this project in case future extensions want to integrate some functionality from \texttt{scrutiny} into it.

%\subsection{Library approach}
%Another improvement worth mentioning is that the tool is divided into modules that provide separate  features such as parsing profiles or visualization. The classes used in the tool can be called from other environments supporting Python programming language by importing them. Examples of such environments are Jupyter\footnote{\url{https://jupyter.org}} based notebooks. An illustration of the usage of such a notebook is a part of the results of this thesis.
