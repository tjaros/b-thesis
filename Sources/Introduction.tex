\chapter{Introduction}
The Trusted Platform Module (TPM) is a device designed to enhance the security of computing platforms. It can be implemented in both hardware and software to provide a broad range of capabilities to the associated platform, including secure storage of cryptographic material and a variety of cryptographic operations.

The TPMs are produced by many vendors adhering to standardized specifications, which define both the mandatory and the optional functional blocks. This results in TPMs having the required functionality in common, but the optional functionality is left to the TPM architects to include. Thus the actual implementations of TPMs can differ both in the supported functionality and performance.

The objective of this thesis is to implement a tool for the visualization of data collected from TPMs. The resulting tool should be able to create visualizations for several datasets which report on supported functionality, performance characteristics, and cryptographic properties. These visualizations should also be compatible with datasets that report on similar information collected from different secure devices, the JavaCard smart cards.

The first chapter introduces the TPM technology and the concepts necessary to understand in the context of TPMs. It also provides insights into several TPM use cases. And lastly, it places this work in the context of secure hardware performance and capabilities testing. The second chapter analyses the current state of the existing solution for the visualization of data collected from TPMs and JavaCard smart cards. Afterward, the design decisions behind the tool created as the objective of this thesis are described. The final chapter presents the outcome of this thesis, a tool called \texttt{AlgTest pyProcess}.
