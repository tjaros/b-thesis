\chapter{State of the Art}

\section{Trusted Platform Module}
The Trusted Platform Module is a system component used as a cryptographic co-processor. It was developed by and standardized by Trusted Computing Group (TCG) consortium with the purpose of laying a foundation on which secure systems could be further created and developed. 

\subsection{History}
The first broadly used specification was TPM 1.1b released in 2003. TPMs released under this specification already provided some basic functions found in modern TPMs consisting of key generation of RSA key pairs, storage, secure authorisation, and device-health attestation. For the sake of assuring privacy the
use of anonymous identity keys based on certificates was introduced. In order to take advantage of such
functionality, these certificates needed to be provided with the TPM and any generation of such keys was available only after owner authorisation. To be able to anonymously prove origin of the keys generated
by TPM a \texttt{privacy certification authority} was created. The integrity of mesurements colected during systems boot sequence is provided by Platform Configuration Registers (PCRs). Both PCRs and identity keys might have been used to prove the health of system's boot sequence~\cite{arthur2015practical}.


The hardware specification was not standardized in TPM 1.1b. This caused various incompatibilities. The TPMs across different vendors provided differing interfaces which in turn required different drivers. Pin-outs on the chips were not prescribed by any standard. Additionally there were no countermeasures against dictionary attacks~\cite{arthur2015practical}.
% Should i also mention DAA ?

With the version 1.2 of the TPM specification additional design decisions needed to take place. From 2005 to 2009, while being in development, it went through numerous releases. With regards to the need of storing shipped certificates for TPM's endorsement keys on hard disk, an about of 2KB of non-volatile RAM was added. In order to support key migration between different TPMs a new design needed to be made, because the old design of key migration would in TPM 1.12 require user to have TPM owner authorisation. The new idea made user able to create migratable keys and then rely on designated third party which could exclusively migrate such keys. Said keys could also be certified thus they were called Certified Migratable keys. Additionally an internal timer able to synchronize with the external one was added in 1.12 which has its use when signing data due to the use of timestamps. Version 1.12 required API to provide backward compatibility for 1.1b. This increased the complexity of the new specification. The TPM 1.12 became widely used in x86 personal computers starting 2005 and later on in 2008 also in servers~\cite{arthur2015practical}.

One of the factors which contributed to the need for yet another specification after TPM 1.12 was the fact that in 2005 there were found some substantial collision attacks against SHA-1 hash function. An analysis with regards to the use of SHA-1 in TPM revealed the attacks not being applicable~\cite{tcg_tpm1.12_sha-1_uses}. Due to the extensive use of SHA-1 in TPM 1.12, the new specification had to permit the use of any hashing algorithm without the need to make any changes to the specification. A so called \texttt{digest agility}. Another problem was the lack of symmetric algorithm to be required in the TPM specification. The use of RSA for encryption of serialized data was impractical, because RSA operations are slow. Neither would help to support bigger sized RSA keys, because that would cause higher cost of the chip, incompatibility issues, and lower performance. Thats why it was decided that in the next specification would adopt support for symmetric encryption, which is faster and more suitable for encryption of large byte streams. Having this many problems, an overhaul of specification would be convenient. And the architects of TPM 2.0 took advantage of the situatio~\cite{arthur2015practical}.


\subsection{Features}


\section{Performance analysis}
\subsection{tpm2-algtest}
The \texttt{tpm2-algtest}\footnote{https://github.com/crocs-muni/tpm2-algtest} is a tool for automatic gathering of information about the TPM2 devices \cite{Struk2019thesis}. The tool uses libraries implementing Trusted Computing Group's TPM2 Software Stack\footnote{https://github.com/tpm2-software/tpm2-tss} which allows for simplification of development when programming applications supposed to interact with the TPM. The tool tests for support of specific commands and supporting routines, values of structures defined in the TPM 2.0 specification \cite{tcg_p3_commands, tcg_p4_supproutines, tcg_p2_structures}. Supported cryptographic algorithms are also a subject to performance analysis where the time to execute such algorithm is repeatedly measured and recorded. Additionally tool uses the TPM to generate key pairs for RSA and ECC based algorithms, so they can be further analysed by various means.
